\documentclass{beamer} % "Beamer" is a word used in Germany to mean video projector. 

\usetheme{metropolis} % Search online for beamer themes to find your favorite or use the Berkeley theme as in this file.


\usepackage{color} % It may be necessary to set PCTeX or whatever program you are using to output a .pdf instead of a .dvi file in order to see color on your screen.
\usepackage{graphicx} % This package is needed if you wish to include external image files.

\theoremstyle{definition} % See Lesson Three of the LaTeX Manual for more on this kind of "proclamation."
\newtheorem*{defn}{Definition} 
\newtheorem{thm}{Theorem}  
\newtheorem{lem}{Lemma}  



\usepackage{multicol}
\usepackage{mathtools}
\usepackage{amsmath}
\usepackage{mathrsfs}
\usepackage{units}
\usepackage{tikz}
\DeclarePairedDelimiter\floor{\lfloor}{\rfloor}
\DeclarePairedDelimiter\ceil{\lceil}{\rceil}
\DeclarePairedDelimiter\gen{\langle}{\rangle}
\DeclarePairedDelimiter\size{|}{|}
\DeclarePairedDelimiter\abs{|}{|}
\DeclarePairedDelimiter\parens{(}{)}
\setlength{\parindent}{0 pt}
\newcommand{\FF}{\mathcal{F}}
\newcommand{\EE}{\mathcal{E}}
\newcommand{\BB}{\mathcal{B}}
\newcommand{\PP}{\mathcal{P}}
\newcommand{\AAA}{\mathcal{A}}
\newcommand{\II}{\mathcal{I}}


\title{New Static Analysis Techniques to Detect Entropy Failure Vulnerabilities}
\author{Andrew Russell \& Rushi Shah} 
\institute{The University of Texas at Austin}
%\date{} 
% Remove the % from the previous line and change the date if you want a particular date to be displayed; otherwise, today's date is displayed by default.

\AtBeginSection[]  % The commands within the following {} will be executed at the start of each section.
{
\begin{frame} % Within each "frame" there will be one or more "slides."  
\frametitle{Presentation Outline} % This is the title of the outline.
\tableofcontents[currentsection]  % This will display the table of contents and highlight the current section.
\end{frame}
} % Do not include the preceding set of commands if you prefer not to have a recurring outline displayed during your presentation.

\begin{document}


\begin{frame}
\titlepage
\end{frame}

\begin{frame}
\frametitle{Entropy Failures: A Historical Perspective}

\begin{center}
\Large{Cryptography \& Digital signature schemes}

\end{center}

\end{frame}

\begin{frame}

\frametitle{Digital Signature Schemes}

\begin{itemize}
\item Canonical cryptographic protocols = encryption and decryption = \emph{data secrecy}
\pause
\item Digital signature schemes = \emph{data authenticity} = who sent the data
\end{itemize}

\begin{center}


\end{center}
\end{frame}

\begin{frame}

\frametitle{Scheme procedures}

\begin{center}
\end{center}


\end{frame}

\begin{frame}

\frametitle{Signature generation}

\textbf{Signature generation} produces a public key $p$ (an identity) and a secret key (a signature) $s$

\begin{center}
\end{center}

\end{frame}

\begin{frame}

\frametitle{Signing}

\textbf{Signing} takes a message $\textcolor{red}{m}$ and produces a signed message $\textcolor{blue}{m'}$

\begin{center}
\end{center}

\end{frame}

\begin{frame}

\frametitle{Verification}

\textbf{Verification} determines if a message $\textcolor{red}{m}$ and a signed message $\textcolor{blue}{m'}$ came from a particular person (designated by their public key $p$)

\begin{center}
\end{center}

\end{frame}

\begin{frame}

\frametitle{Real world}

This is used billions of times per day

\begin{center}

vs. 

\end{center}

\end{frame}

\begin{frame}
\frametitle{Brief recap}

Digital signature schemes provide \emph{authenticity}, not secrecy

Still very necessary for any good cryptosystem

\end{frame}


\begin{frame}
\frametitle{\;}

\begin{center}
\Large{Elliptic Curves}
\end{center}

\end{frame}

\begin{frame}

\frametitle{Defining elliptic curves}
\begin{definition}
An elliptic curve over a field $\mathbb{F}$ consists of the points $(x,y)$ given by the equation
\[
	y^2 = x^3 + \textcolor{red}{a}x + \textcolor{blue}{b}
\]
where $\textcolor{red}{a},\textcolor{blue}{b} \in \mathbb{F}$ and $x^3 + \textcolor{red}{a}x + \textcolor{blue}{b}$ has no repeated factors.
\end{definition}

\scriptsize{\emph{(a field is a set where we can add, subtract, multiply and divide except for 0; think $\mathbb{R}$ or $\mathbb{Q}$)}}


\end{frame}

\begin{frame}

\frametitle{Examples}

The curve $\textcolor{red}{a} = -1$ and $\textcolor{blue}{b} = 1$ over $\mathbb{R}$:
                        
\begin{center}
\end{center}

\end{frame}

\begin{frame}

\frametitle{Examples}


The curve $\textcolor{red}{a} = -1$ and $\textcolor{blue}{b} = 1$ over $\mathbb{F}_{163}$:
                        
\begin{center}
\end{center}

\end{frame}

\begin{frame}

\frametitle{Arithmetic}

We can perform arithmetic on the points of the elliptic curve.  To add two points $\textcolor{red}{P}$ and $\textcolor{blue}{Q}$:

\begin{enumerate} 
\item Draw the secant line between the two points
\pause
\item Find the third point of intersection of this line and the curve
\pause
\item Find the reflection of this point across the $x$-axis, and we're done!
\end{enumerate}

\end{frame}

\begin{frame}

\frametitle{Arithmetic example}

\begin{center}
\end{center}

\end{frame}

\begin{frame}

\frametitle{Arithmetic example}

\begin{center}
\end{center}

\end{frame}

\begin{frame}

\frametitle{Arithmetic example}

\begin{center}
\end{center}

\end{frame}

\begin{frame}

\frametitle{Adding a point to itself}

\begin{center}
For an integer $n$, we denote $P$ added to itself $n$ times by $nP$

\end{center}

\end{frame}

\begin{frame}

\frametitle{Adding a point to itself}

\begin{center}
For an integer $n$, we denote $P$ added to itself $n$ times by $nP$

\end{center}

\end{frame}

\begin{frame}

\frametitle{Adding a point to itself}

\begin{center}
For an integer $n$, we denote $P$ added to itself $n$ times by $nP$

\end{center}

\end{frame}

\begin{frame}

\frametitle{Fun fact}

The points of an elliptic curve actually form an abelian \emph{group}.

\end{frame}

\begin{frame}

\frametitle{Bilinear pairings}

We have a function $\phi: E \times E \to E$ such that it is \emph{bilinear} (we can pull out coefficients from either argument):

\begin{center}
$\phi(\textcolor{red}{a}P,Q) = \textcolor{red}{a}\cdot\phi(P,Q)$ 

and 

$\phi(P,\textcolor{blue}{b}Q) = \textcolor{blue}{b}\cdot\phi(P,Q)$
\end{center}

\end{frame}


\begin{frame}
\frametitle{\;}

\begin{center}
\Large{Digital signatures using elliptic curves}
\end{center}

\end{frame}


\begin{frame}

\frametitle{Elliptic curves and a signature scheme}

Some setup: a trusted third party will generate an elliptic curve $E$ and a point $Q \in E$ randomly, and publish both publicly for everyone to see (e.g. via the Internet).

\begin{center}
\end{center}

\end{frame}


\begin{frame}

\frametitle{Signature generation}

Simply generate a random integer $\textcolor{blue}{s} \in \mathbb{Z}$ and compute the elliptic curve point $P = \textcolor{blue}{s}Q$.  

\begin{center}
\end{center}

Your \textbf{secret key} is $\textcolor{blue}{s}$ and your \textbf{public key} is $P$.

\end{frame}

\begin{frame}

\frametitle{Signing}

Let $\textcolor{red}{m} \in \mathbb{Z}$ be a message.  \pause To sign this message, first compute the elliptic curve point $M = \textcolor{red}{m}Q$.  

\begin{center}
\end{center}

Thus, $M' = \textcolor{blue}{s}M = \textcolor{blue}{s}(\textcolor{red}{m}Q)$ is the signed message.
\end{frame}

\begin{frame}

\frametitle{Verification}

To verify, we compute $R_1 = \phi(Q, M')$ and $R_2 = \phi(P, \textcolor{red}{m}Q)$ and test if $R_1 = R_2$.  \pause 

To see why this works:  since $P = \textcolor{blue}{s}Q$ and $M' = \textcolor{blue}{s}(\textcolor{red}{m}Q)$ then by the bilinearity of $\phi$ we have:

\begin{center}
\begin{multicols}{2}
\noindent
\begin{align*}
R_1 &= \phi(Q,M')\\
&= \phi(Q,\textcolor{blue}{s}\textcolor{red}{m}Q)\\
&= \textcolor{blue}{s}\textcolor{red}{m}\cdot\phi(Q,Q)\\\\
\end{align*}

\pause

\begin{align*}
R_2 &= \phi(P, \textcolor{red}{m}Q)\\
&= \phi(\textcolor{blue}{s}Q, \textcolor{red}{m}Q)\\
&= \textcolor{blue}{s}\cdot\phi(Q,\textcolor{red}{m}Q)\\
&= \textcolor{blue}{s}\textcolor{red}{m}\cdot\phi(Q,Q)
\end{align*}
\end{multicols}
\end{center}
\end{frame}


\begin{frame}
\frametitle{\;}

\begin{center}
\Large{Security?}


\tiny{Source: https://xkcd.com/538/}
\end{center}

\end{frame}

\begin{frame}

\frametitle{Attacking our scheme}

Given just a public key $P = sQ$ and public point $Q$, if we could recover the secret key $s$ then we can break the scheme by being able to sign any message we want

\end{frame}

\begin{frame}

\frametitle{Security in modern cryptography}

Modern crypto defines security through reductions to ``hard problems"
\pause

\textbf{Proof of security:} a cryptosystem X is \emph{just as hard} to hack as it is to solve a ``hard" computational problem

\pause

Ex.: \textbf{\emph{If}} you can factor large integers quickly, \textbf{\emph{then}} you can break the RSA encryption scheme
\end{frame}

\begin{frame}

\frametitle{Defining our hard problem}

We can make an analogy to computing logarithms: $P = Q^s$ and $Q$ 

(because the points of an elliptic curve form a group, our $sQ = Q + Q + \cdots + Q$ could be written multiplicatively as $Q^s = Q \cdot Q \cdots Q$)

\pause 

So, we want to take the ``logarithm" of both sides to recover $s$: $$\log_Q(P) = s$$
\end{frame}

\begin{frame}

\frametitle{Discrete log problems}

This is easy, with say, $\mathbb{F} = \mathbb{R}$

\pause

It is very hard in a discrete setting, with $\mathbb{F} = \mathbb{Z}/p\mathbb{Z}$

\pause

This is known as the discrete log problem
\end{frame}

\begin{frame}

\frametitle{The discrete log problem}

Discrete Log Problem: given two elliptic curve points $R = xP$ and $P$ over a \emph{finite field}, we cannot easily compute $x \in \mathbb{Z}$
\end{frame}

\begin{frame}

\frametitle{Recap}

We've covered:

\begin{itemize}
\item Basic elliptic curve arithmetic
\item Digital signature schemes from EC machinery (particularly, pairing)
\item A canonical ``proof of security" for why this scheme is secure
\end{itemize}

My thesis is primarily on the \emph{pairing} portion (specifically, the Weil pairing)
\end{frame}

\begin{frame}

\frametitle{The End}

\begin{center}
\huge{Questions?}

\small{and thank you!}
\end{center}

\end{frame}


\end{document}