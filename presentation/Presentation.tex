\documentclass{beamer} % "Beamer" is a word used in Germany to mean video projector. 

\usetheme[titleformat=allcaps]{metropolis} % Search online for beamer themes to find your favorite or use the Berkeley theme as in this file.


\usepackage{color} % It may be necessary to set PCTeX or whatever program you are using to output a .pdf instead of a .dvi file in order to see color on your screen.
\usepackage{graphicx} % This package is needed if you wish to include external image files.

\theoremstyle{definition} % See Lesson Three of the LaTeX Manual for more on this kind of "proclamation."
\newtheorem*{defn}{Definition} 
\newtheorem{thm}{Theorem}  
\newtheorem{lem}{Lemma}  

\usepackage{proof}
\usepackage{amssymb}
\usepackage{amsmath}
\usepackage{amsthm}
\usepackage{mathtools}
\usepackage{algorithm}
\usepackage{algorithmicx}
\usepackage{algpseudocode}
\usepackage{stmaryrd}

\newcommand{\cross}{\otimes{}}

\usepackage{multicol}
\usepackage{mathtools}
\usepackage{amsmath}
\usepackage{mathrsfs}
\usepackage{units}
\usepackage{tikz}
\DeclarePairedDelimiter\floor{\lfloor}{\rfloor}
\DeclarePairedDelimiter\ceil{\lceil}{\rceil}
\DeclarePairedDelimiter\gen{\langle}{\rangle}
\DeclarePairedDelimiter\size{|}{|}
\DeclarePairedDelimiter\abs{|}{|}
\DeclarePairedDelimiter\parens{(}{)}
\setlength{\parindent}{0 pt}
\newcommand{\FF}{\mathcal{F}}
\newcommand{\EE}{\mathcal{E}}
\newcommand{\BB}{\mathcal{B}}
\newcommand{\PP}{\mathcal{P}}
\newcommand{\AAA}{\mathcal{A}}
\newcommand{\II}{\mathcal{I}}


\title{New Static Analysis Techniques to Detect Entropy Failure Vulnerabilities}
\author{Andrew Russell \& Rushi Shah} 
\institute{The University of Texas at Austin}
%\date{} 
% Remove the % from the previous line and change the date if you want a particular date to be displayed; otherwise, today's date is displayed by default.

\AtBeginSection[]  % The commands within the following {} will be executed at the start of each section.
{
\begin{frame} % Within each "frame" there will be one or more "slides."  
\frametitle{Presentation Outline} % This is the title of the outline.
\tableofcontents[currentsection]  % This will display the table of contents and highlight the current section.
\end{frame}
} % Do not include the preceding set of commands if you prefer not to have a recurring outline displayed during your presentation.

\begin{document}


\begin{frame}
\titlepage
\end{frame}

\begin{frame}
\frametitle{Entropy Failures in the Wild}
     \begin{itemize} 
        \item Debian OpenSSL (2008)
        \begin{itemize}
            \item Removed key seed function call due to Valgrind error
        \end{itemize}

        \pause

        \item FreeBSD (2016)
        \begin{itemize}
            \item Kernel randomness never switches to ``secure mode'' after boot
       \end{itemize}
	 \end{itemize}
	 % Visual explanation? 
	 % Definition of entropy? What is entropy? System events. What is not entropy? getPID(). 
\end{frame}

\begin{frame}
\frametitle{How Could This Have Been Avoided?}
\begin{itemize}
    \item Code audits? Too much \$\$
    \item Taint analysis: Works, but overapproximates
    \item Idea: Use version control history to get better answers
    \item Audit code once, then use static analysis to prove you haven't introduced bugs with small changes to program
    \begin{itemize}
        \item Key idea: Prove version 2 is secure relative to version 1.
    \end{itemize}

	\item ``Differential" approach works well with way software is written in the real world (CI, etc.)
\end{itemize}
\end{frame}

\begin{frame}
\frametitle{Problem Statement}

	Given two versions $P_1, P_2$ of a program, prove that if legitimate sources of entropy in $P_1$ flow into their sinks properly, then the same is true of $P_2$. 

	If $\tau_1$ is the taint set of variable $x$ passed to sink in $P_1$ and $\tau_2$ is the taint set of $x$ in $P_2$, then we would like \[\tau_1 == \tau_2\]
	

	% In this case, a sink is basically a function that accepts a parameter that is expected to be sufficiently entropic

\end{frame}

\begin{frame}
    \frametitle{Our Contribution}
    
        We propose a novel algorithm using recent developments in static analysis to solve this problem in a tractable way.
        
    
        % In this case, a sink is basically a function that accepts a parameter that is expected to be sufficiently entropic
    
    \end{frame}

\begin{frame}
\frametitle{Safety \& Static Analysis}

\begin{itemize}
    \item $2$-safety property: making an assertion based on two runs of a program (ex: symmetry --- $P(x,y) = P(y,x)$)
    \item Static analysis technique called ``predicate abstraction" can prove $1$-safety properties
    \item Existing techniques can reduce $2$-safety properties into $1$-safety properties via ``product programs"
\end{itemize}

	% Transform by sequentially composing or with product program. Can we combine these two approach with domain specific knowledge to make this a more tractable problem? 

	% We'll say a few words about existing techniques, how they relate to this specific domain, and possible domain-specific optimizations we can make

\end{frame}

\begin{frame}
    \frametitle{Product Programs}
    
    \begin{itemize}
        \item Consider our symmetry example: \[a := P(x,y) ; b := P(y,x) ; \textsf{assert}(a ==b)\]
        \item However, it's hard to prove things about these sequentially composed programs

        \pause
        \item Instead, form product program $P_1 \times P_2 \equiv P_1\ ;\ P_2$ 

        \begin{itemize}
            \item ``Synchronized'' in a way that makes things easier for the verifier
        \end{itemize}
    \end{itemize}
    
    
    \end{frame}

    \begin{frame}
        \frametitle{Product Programs}
        
        \begin{center}
        \begin{columns}
            \begin{column}{0.48\textwidth}
            $$P_1$$

                if (p):
                    x $\gets$ 1;
                else:
                    x $\gets$ 2;
% \begin{verbatim}
% if f():
%     x = 1;
% else:
%     x = 2;
% \end{verbatim}
            \end{column}
            \begin{column}{0.48\textwidth}
            $$P_2$$ 

                if (p):
                    x $\gets$ 2;
                else:
                    x $\gets$ 1;
% \begin{verbatim}
% if f():
%     x = 2;
% else:
%     x = 1;
% \end{verbatim}
            \end{column}
        \end{columns}
    \end{center}
    \end{frame}

    \begin{frame}
        \frametitle{Product Programs}   
        $$P_1\ ;\ P_2$$

        \begin{center}
            \begin{algorithm}[H]
                \begin{algorithmic}
                    \If {$p$}
                        \State $x_1 \gets 1$
                    \Else
                        \State $x_1 \gets 2$
                    \EndIf
                    \If {$p$}
                        \State $x_2 \gets 2$
                    \Else
                        \State $x_2 \gets 1$
                    \EndIf
                    \State \textsf{assert}$(x_1 == x_2)$
                \end{algorithmic}
            \end{algorithm}
    \end{center}
        
        \end{frame}

        \begin{frame}
            \frametitle{Product Programs}   
            $$P_1\times P_2$$

        \begin{center}
            \begin{algorithm}[H]
                \begin{algorithmic}
                    \If {$p$}
                        \State $x_1 \gets 1$
                        \State $x_2 \gets 2$
                    \Else
                        \State $x_1 \gets 2$
                        \State $x_2 \gets 1$
                    \EndIf
                    \State \textsf{assert}$(x_1 == x_2)$
                \end{algorithmic}
            \end{algorithm}
        \end{center}
            
            \end{frame}

\begin{frame}
\frametitle{Approach: High-level overview}

	\begin{enumerate}
		\item Instrumentation: Convert taint analysis problem to safety problem
		\item Product Program: Convert $2$-safety property into $1$-safety property
		\item Use off-the-shelf static analysis tool to verify resulting program
	\end{enumerate}

\end{frame}

\begin{frame}
    \frametitle{Language Semantics}
    
        \[
            \begin{array}{l l c l}
                Statement & S & := & 
                    \ \ \mathsf{Atom} \\
                    & & & |~ S_1\ ;\ S_2 \\
                    & & & |~ \text{if }p\text{ then } S_1 \text{ else } S_2 \\
                    & & & |~ \text{while }p\text{ } S \\
                Predicate & p &:=& \top ~~|~~ \bot ~~|~~ \mathsf{Atom} ~~|~~ \lnot p ~~|~~ p \odot p \\
                Operator & \odot &:=& \land ~~|~~ \lor \\
            \end{array}
        \]
    
        % Say in words we can model C with these semantics 
    
    \end{frame}

\begin{frame}
    \frametitle{Synchronized Product Program}
    
       \begin{itemize}
           \item Easiest to reason about
           \item However: exponential blowup (in number of branches)
           \pause
           \item Basic inference rules:
       \end{itemize}

    
        \[
            \begin{array}{c}
                \infer[]{A_1 \cross A_2 \leadsto A_1\ ;\ A_2}{} \\ \\
                \infer[]{S_1 \cross S_2 \leadsto P}{
                    S_2 \cross S_1 \leadsto P
                } 
            \end{array}
        \]
        \pause
        \[
            \begin{array}{c}
                \infer[]{if(p)\ then\ S_1\ else\ S_2 \cross S \leadsto P}{
                    \text{$S_1 \cross S \leadsto S_1'$} &
                    \text{$S_2 \cross S \leadsto S_2'$} &
                    \text{$P = if(p)\ then\ S_1'\ else\ S_2'$}
                } \\ \\
                \pause
                \infer[]{while(p_1)\ S_1 \cross while(p_2)\ S_2 \leadsto P_0\ ;\ P_1\ ;\ P_2}{
                    \text{$P_0 = while(p_1 \land p_2)\ S_1\ ;\ S_2$} &
                    \text{$P_1 = while(p_1)\ S_1$} &
                    \text{$P_2 = while(p_2)\ S_2$}
                }
            \end{array}
        \]
        
    \end{frame}

    \begin{frame}
        \frametitle{Synchronized Product Program}
        $$P_1 \times P_2$$
        \begin{center}
            \begin{minipage}{0.5\linewidth}
            
            \begin{algorithm}[H]
                \begin{algorithmic}
                    \If {$p$}
                        \State $x_1 \gets 1$
                        \If {$p$}
                            \State $x_2 \gets 2$
                        \Else
                            \State $x_2 \gets 1$
                        \EndIf
                    \Else
                        \State $\vdots$
                    \EndIf
                    \State \textsf{assert}$(x_1 == x_2)$
                \end{algorithmic}
            \end{algorithm}
        \end{minipage}
        \end{center}
        
    \end{frame}

    \begin{frame}
        \frametitle{Hybrid Product Program}
        
        \begin{itemize}
            \item Key insight: don't reason precisely about unrelated parts of the program
            \item ``Unrelated'' if not tainted. Use environment $\Gamma$ and add following inference rule:
            \[
                \begin{array}{c}
                    \infer[]{\Gamma \vdash S_1 \cross S_2 \leadsto S_1\ ;\ S_2}{
                        \Gamma \nvdash S_1 \\
                        \Gamma \nvdash S_2
                    } \\
                \end{array}
            \]
        \end{itemize}
            
        \end{frame}

% only reason to mention instrumentation is to make it clear how we replace the sources with labelled constants and what we assert at the sinks


\begin{frame}
\frametitle{Instrumentation}
    
\begin{itemize}
    \item Replace sources with labelled constants
    \item For values that are tainted by more than one source (for example $S_1 + S_2$) replace with one of two uninterpreted functions over the sources: \begin{enumerate}
		\item $\textsf{preserving}(s_1, s_2, \ldots, s_n)$. Ex: $+, \oplus$
		\item $\textsf{non-preserving}(s_1, s_2, \ldots, s_n)$. Ex: $<<, >>$
    \end{enumerate}
    \item Perform taint analysis on sources to generate environment $\Gamma$ which marks statements involving tainted variables.
\end{itemize}

\end{frame}

\begin{frame}
\frametitle{Taint $\to$ Safety}
	
	For every variable $x$ that is tainted in a statement $s$ that is marked as a sink, insert an assertion: \[\textsf{assert}(x_1 == x_2)\]

    Recall we replaced sources with labelled constants and propagated them, so this will be asserting the taintsets of the two variables are equivalent.
    
    \pause

    \begin{itemize}
        \item If the assertion can be statically verified, $P_2$ is correct modulo $P_1$
    \end{itemize}
	

\end{frame}

\begin{frame}
\frametitle{Acknowledgements}
	\[\text{Prof. Hovav Shacham and Prof. Işil Dillig}\]
\end{frame}

\begin{frame}
    \begin{center}
    Thank you! 
    
    \LARGE{Questions?}
    \end{center}
\end{frame}


% \end{frame}

% \begin{frame}
% \frametitle{Entropy Failures: A Historical Perspective}

% 	\begin{center}
% 	\Large{
% 	\begin{enumerate}
% 		\item OpenSSL
% 		\item FreeBSD
% 	\end{enumerate}
% 	}
% 	\end{center}

% 	% Visual explanation?

% \end{frame}

% \begin{frame}
% \frametitle{How Could This Have Been Avoided?}

% 	- Audit code once, then use relational verification to prove you haven't introduced bugs with small changes to program. 

% 	- This differential approach works well with the way software is written (CI, etc.)

% \end{frame}

% \begin{frame}
% 	\[\text{\Large{Background/Initial Approach}}\]
% \end{frame}

% \begin{frame}
% \frametitle{Language Semantics}

% 	\[
% 		\begin{array}{l l c l}
% 			Statement & S & := & 
% 				\ \ A \\
% 				& & & |~ S_1\ ;\ S_2 \\
% 				& & & |~ \text{if }p\text{ then } S_1 \text{ else } S_2 \\
% 				& & & |~ \text{while }p\text{ } S \\
% 			Predicate & p &:=& \top ~~|~~ \bot ~~|~~ A ~~|~~ \lnot p ~~|~~ p \odot p \\
% 			Operator & \odot &:=& \land ~~|~~ \lor \\
% 		\end{array}
% 	\]

% 	% Say in words we can model C with these semantics 

% \end{frame}

% \begin{frame}
% \frametitle{Taint Analysis}

% 	- Definition, terminology, uses, etc.

% 	- Sources are legitimate sources of entropy (/dev/random/)

% 	- Sinks are things like cryptographic algorithms (KDF)

% 	- But taint analysis is unsound when used on two versions of the program (overapproximation)

% 	% Couple of slides instead of just one

% \end{frame}

% \begin{frame}
% \frametitle{Predicate Abstraction}

% 	- Finer grained version of taint analysis across versions of a program.

% 	- Taint set of variable in program two should be a superset of taint set of variable in program one.

% 	\pause

% 	- Weaker version: if $v_1$ is the taint set of variable $v$ passed to sink in program one and $v_2$ is the taint set of $v$ in program two, then we would like \[assert(v_1 == v_2)\]

% \end{frame}

% \begin{frame}
% \frametitle{Predicate Abstraction}

% 	- Off-the-shelf state-of-the-art: CPAChecker

% \end{frame}

% \begin{frame}
% \frametitle{Product Programs}

% 	- Sequential Composition \[S_1\ ;\ S_2\]

% 	\pause

% 	- Synchronized Composition \[S_1 \cross S_2\]

% 	\pause

% 	- Hybrid?

% \end{frame}

% \begin{frame}
% 	\[\text{\Large{Algorithm}}\]
% \end{frame}

% \begin{frame}
% \frametitle{High level overview}

% 	\begin{enumerate}
% 		\item Instrumentation
% 		\item Product Program
% 		\item Assertions + CPAChecker
% 	\end{enumerate}

% \end{frame}

% \begin{frame}
% \frametitle{Instrumentation}
	
% 	- Replace sources with labelled constants

% 	- Perform taint analysis on sources to generate environment $\Gamma$ which marks statements involving tainted variables. 

% \end{frame}

% \begin{frame}
% \frametitle{Instrumentation}
	
% 	- For values that are tainted by more than one source (for example $S_1 + S_2$) replace with one of two uninterpreted functions over the sources: \begin{enumerate}
% 		\item $preserving(s_1, s_2, \ldots, s_n)$
% 		\item $nonPreserving(s_1, s_2, \ldots, s_n)$
% 	\end{enumerate}
	
% 	- Preserving functions are +, XOR, etc.
	
% 	- Non-preserving functions are left or right shift, etc.

% \end{frame}

% \begin{frame}
% \frametitle{Sequential Product Program}
	
% \end{frame}

% \begin{frame}
% \frametitle{Naive Synchronized Product Program}
	
% \end{frame}

% \begin{frame}
% \frametitle{Heuristic-Optimized Synchronized Product Program}
	
% \end{frame}

% \begin{frame}
% \frametitle{Assertions + CPAChecker}
	
% \end{frame}

% \begin{frame}
% \frametitle{Correctness}
	
% \end{frame}

% \begin{frame}
% 	\[\text{\Large{Future Work}}\]
% \end{frame}

% \begin{frame}
% \frametitle{Implementation Evaluation}
	
% \end{frame}

% \begin{frame}
% 	\[\text{\Large{Conclusion}}\]
% \end{frame}

% \begin{frame}
% \frametitle{Acknowledgements}
% 	\[\text{Prof. Hovav Shacham and Prof. Isil Dillig}\]
% \end{frame}

% \begin{frame}
% \frametitle{References}
	
% \end{frame}

\end{document}