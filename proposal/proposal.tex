% TEMPLATE for Usenix papers, specifically to meet requirements of
%  USENIX '05
% originally a template for producing IEEE-format articles using LaTeX.
%   written by Matthew Ward, CS Department, Worcester Polytechnic Institute.
% adapted by David Beazley for his excellent SWIG paper in Proceedings,
%   Tcl 96
% turned into a smartass generic template by De Clarke, with thanks to
%   both the above pioneers
% use at your own risk.  Complaints to /dev/null.
% make it two column with no page numbering, default is 10 point

% Munged by Fred Douglis <douglis@research.att.com> 10/97 to separate
% the .sty file from the LaTeX source template, so that people can
% more easily include the .sty file into an existing document.  Also
% changed to more closely follow the style guidelines as represented
% by the Word sample file. 

% Note that since 2010, USENIX does  not require endnotes. If you want
% foot of page notes, don't include the endnotes package in the 
% usepackage command, below.

% This version uses the latex2e styles, not the very ancient 2.09 stuff.
\documentclass[letterpaper,twocolumn,10pt]{article}
\usepackage{usenix,epsfig,endnotes}
\PassOptionsToPackage{hyphens}{url}\usepackage{hyperref}
\begin{document}

%don't want date printed
\date{}

%make title bold and 14 pt font (Latex default is non-bold, 16 pt)
\title{\Large \bf CS380S: Project Proposal}

%for single author (just remove % characters)
\author{
{\rm Rushi Shah}
\and
{\rm Andrew Russell}
}

\maketitle

\section{Project Idea}

\subsection{Motivation}
One common misuse of cryptography is the misuse of entropy. Without proper random inputs, many cryptographic algorithms
are vulnerable to basic forms of cryptanalysis. Some cryptographic schemes, such as DSA, can even disclose long-term secrets
like the signing key when the random per-message input is low entropy, made public, or nonunique.

We have two motivating examples. First, we would like our tool to be able to detect the issue with the Debian/OpenSSL pseudo-random number generator that was exposed in 2008. 
Second, we would like to identify potential vulnerabilities in current cryptocurrency wallet code as many cryptocurrency protocols rely on DSA.

\subsection{Goals}
We plan to use data dependency tools to determine how entropic inputs in a given program are used by various cryptographic algorithms. 
That will allow us to identify if and when entropy is too low or is misused. We plan to either produce
this as a code integration tool for developers to use as part of a compiler toolchain, or use this tool to analyze a large number of codebases
found ``in the wild,'' such as those written by amateurs.

Our tool will seek to generate a human-checkable dependency graph from source code
using taint analysis on functions and function inputs manually specified via annotation by the developer.
This graph will allow the developer to manually verify that entropy is used properly; a stretch goal 
of ours would be to automate the identification of a subset of known misuses of entropy.

\section{Rough Plan}

\begin{enumerate}
	\item Determine data dependency tool (for C/C++) to use, conduct background research (\textit{2 weeks})
	\item Adapt tool to identify the OpenSSL bug (\textit{5 weeks})
	\item Test on other cases like Bitcoin wallets (\textit{3 weeks})
	\item Prepare presentation/writeup (\textit{1 week})
	\item Integrate tool into compiler toolchain like clang (\textit{optional})
\end{enumerate}

\section{Research Hypotheses}

We principally have two hypotheses we would like to test.

\begin{enumerate}
	\item An automated tool can detect entropy bugs in real-world programs.
	\item Entropy is insufficiently propagated in programs that rely on cryptography.
\end{enumerate}

\section{Related Work}

Static code analysis has a history of identifying security vulnerabilities at a source code level.
Some examples include SQL injection, cross-site scripting exploits, and buffer overflow attacks. However,
there has not been any attempt in the literature to statically analyze source code for cryptographic
vulnerabilities stemming from entropy misuse, which we seek to do.

% TODO maybe we need to say what exactly we found in the links instead of just dropping links?

\begin{enumerate}
	\item Debian/OpenSSL Bug 
		\begin{enumerate}
			\item \url{https://www.schneier.com/blog/archives/2008/05/random_number_b.html}
			\item \url{https://research.swtch.com/openssl}
			\item \url{https://freedom-to-tinker.com/2013/09/20/software-transparency-debian-openssl-bug/}
			\item \url{https://www.cs.umd.edu/class/fall2017/cmsc818O/papers/private-keys-public.pdf}
		\end{enumerate}
	\item Data flow
		\begin{enumerate}
			\item \url{https://en.wikipedia.org/wiki/Data-flow_analysis}
			\item \url{https://www.seas.harvard.edu/courses/cs252/2011sp/slides/Lec02-Dataflow.pdf}
		\end{enumerate}
	\item Static Program Analysis
		\begin{enumerate}
            \item \url{https://cs.au.dk/~amoeller/spa/spa.pdf}
            \item \url{https://ieeexplore.ieee.org/stamp/stamp.jsp?arnumber=6859783}
		\end{enumerate}

	% TODO maybe have a section about cryptocurrency wallets, etc.
\end{enumerate}


% TODO

\end{document}
