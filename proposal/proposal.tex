% TEMPLATE for Usenix papers, specifically to meet requirements of
%  USENIX '05
% originally a template for producing IEEE-format articles using LaTeX.
%   written by Matthew Ward, CS Department, Worcester Polytechnic Institute.
% adapted by David Beazley for his excellent SWIG paper in Proceedings,
%   Tcl 96
% turned into a smartass generic template by De Clarke, with thanks to
%   both the above pioneers
% use at your own risk.  Complaints to /dev/null.
% make it two column with no page numbering, default is 10 point

% Munged by Fred Douglis <douglis@research.att.com> 10/97 to separate
% the .sty file from the LaTeX source template, so that people can
% more easily include the .sty file into an existing document.  Also
% changed to more closely follow the style guidelines as represented
% by the Word sample file. 

% Note that since 2010, USENIX does not require endnotes. If you want
% foot of page notes, don't include the endnotes package in the 
% usepackage command, below.

% This version uses the latex2e styles, not the very ancient 2.09 stuff.
\documentclass[letterpaper,twocolumn,10pt]{article}
\usepackage{usenix,epsfig,endnotes}
\PassOptionsToPackage{hyphens}{url}\usepackage{hyperref}
\begin{document}

%don't want date printed
\date{}

%make title bold and 14 pt font (Latex default is non-bold, 16 pt)
\title{\Large \bf CS380S: Project Proposal}

%for single author (just remove % characters)
\author{
{\rm Rushi Shah}
\and
{\rm Andrew Russell}
}

\maketitle

\section{Project Idea}

We plan to use Data Dependency tools to determine how system entropy in a given program is used by various cryptographic algorithms. That will allow us to identify if and when entropy is too low or is misused. We can release this as a tool for developers to use as part of a compiler toolchain. We can also use the tool on current projects that might be misusing crypto/entropy. 

We have two motivating examples. First, we would like our tool to be able to detect the issue with the Debian/OpenSSL pseudo-random number generator that was exposed in 2008. Second, we would like to identify potential vulnerabilities in current cryptocurrency wallet code. 

\section{Rough Plan}

% TODO

\section{Research Hypothesis}

% TODO

\section{Related Work}

% TODO maybe we need to say what exactly we found in the links instead of just dropping links?
\subsection{Background information}

\begin{enumerate}
	\item Debian/OpenSSL Bug 
		\begin{enumerate}
			\item \url{https://www.schneier.com/blog/archives/2008/05/random_number_b.html}
			\item \url{https://research.swtch.com/openssl}
			\item \url{https://freedom-to-tinker.com/2013/09/20/software-transparency-debian-openssl-bug/}
			\item \url{https://www.cs.umd.edu/class/fall2017/cmsc818O/papers/private-keys-public.pdf}
		\end{enumerate}
	\item Data flow
		\begin{enumerate}
			\item \url{https://en.wikipedia.org/wiki/Data-flow_analysis}
			\item \url{https://www.seas.harvard.edu/courses/cs252/2011sp/slides/Lec02-Dataflow.pdf}
		\end{enumerate}

	% TODO maybe have a section about cryptocurrency wallets, etc.
\end{enumerate}

\subsection{Related Research}

% TODO

\end{document}
